\subsection{The model}
% TODO http://tex.stackexchange.com/a/155317/5966 ?
\begin{figure}
\centering
\begin{tikzpicture}[scale=1.1] % TODO SCALE!!!
\input{pic/resonator_bundle.tex}
\end{tikzpicture}
\caption{Квантовый граф $\Gamma$, состоящий из полубесконечных ребер $\Omega_L, \Omega_R$ и рассеивателя $\Omega$, представляющего из себя $W$ идентичных ребер длины 1. В вершинах $V_a$, $V_b$ находятся $\delta$-образные потенциальные ямы глубины $a$ и $b$ соответственно.}\label{fig:res_bundle}
\end{figure}

Let's investigate the model on \autoref{fig:res_bundle}. Schrodinger's operator is defined on edges of the graph as  $-\dv[2]{\psi}{x}$. At the vertices $V_a$ and $V_b$ we require the wavefunction and its derivatives to be continuous:

\begin{equation}\label{eq:bundle_system}
\begin{aligned}
   \forall i: \psi_i(0) &= \psi_L(0)
\\ \forall i: \psi_i(1) &= \psi_R(0)
\\ \psi'_L(0) - \sum\limits_{i = 1}^W \psi'_i(0) &= 0
\\ -\psi'_R(0) + \sum\limits_{i = 1}^W \psi'_i(1) &= 0
\end{aligned}
\end{equation}

\subsection{Calculation of $S$ matrix}
After solving \ref{eq:bundle_system} в сочетании с \ref{eq:wlwl}, we get:
\begin{equation*}
\det S(k) = \frac{2 i \, W \cos\left(k\right) - {\left(W^{2} + 1\right)} \sin\left(k\right)}{2 i \, W \cos\left(k\right) + {\left(W^{2} + 1\right)} \sin\left(k\right)}
\end{equation*}


\begin{equation}\label{eq:bundle_s}
s(k) = \abs{\det S(k)} = \abs{\frac{2 i \, W \cos\left(k\right) - {\left(W^{2} + 1\right)} \sin\left(k\right)}{2 i \, W \cos\left(k\right) + {\left(W^{2} + 1\right)} \sin\left(k\right)}}
\end{equation}

\mtodo{график S-матрицы?}

TODO: define contour

Let's fix $r$, and further, for the sake of readability, let's define $C = C(r)$ and $R = R(r)$. For each $r$, we estimate the integral and show that the sequence of estimates converges to zero.

Let's use Cauchy–Schwarz's inequality:
\[
\big| \langle u,v \rangle \big| \leq \left\|u\right\| \left\|v\right\|
\]
, to be precise, it's $\mcL^2[a, b]$ version:
\[
\abs{
\int\limits_{t=a}^{b} f(t) g^*(t) dt
}
\le
\sqrt{\int\limits_{t=a}^b \abs{f(t)}^2 dt }
\sqrt{\int\limits_{t=a}^b \abs{g(t)}^2 dt }
\]
% 
To use inequality, we are going to represent the integrand in (\ref{eq:critp}) as $f(t) g^*(t)$ as follows:
\begin{equation}\label{eq:split}
\begin{aligned}
a      &= 0 \\
b      &= 2 \pi \\
f(t)   &= \ln s(R \eexp{\iu t} + \iu C) \frac{\sqrt{R}}{R \eexp{\iu t} + \iu C + i} \\
g^*(t) &= \frac{\sqrt{R}}{R \eexp{\iu t} + \iu C + i}
\end{aligned}
\end{equation}
Next, we will show that $\norm{g}$ (that is, $\sqrt{\int \abs{g(t)}^2}$) is bounded by a constant which does not depend on $r$, and $\norm{f}$ goes to $0$ as $r$ goes to $1$, hence (\ref{eq:critp}) converges.

\subsubsection{Estimate of $\norm{g}$}

First, we evaluate the integrand:
\begin{align*}
\abs{g(t)}^2 = \abs{g^*(t)}^2
&=   \frac{\sqrt{R}^2}{\abs{R \cos t + \iu R \sin t + \iu C + i}^2} \\
&=   \frac{R}{R^2 \cos^2 t + (R \sin t + C + 1)^2} \\
&= \frac{R}{R^2 \cos^2 t + R^2 \sin^2 t + (C + 1)^2  + 2 R (C + 1) \sin t} \\
&=   \frac{1}{R + \frac{(C + 1)^2}{R} + 2 (C + 1) \sin t} \\ 
\end{align*}

We have to integrate this function on interval $[0, 2 \pi]$. Integral of such type is well-known: (TODO REFERENCE)
\[
\int\limits_{0}^{2 \pi} \frac{dx}{a + b \sin x} = \frac{2 \pi}{\sqrt{a^2 - b^2}}
\]
, when $a > b$. In our case, $a = R + \frac{(C + 1)^2}{R}$, $b = 2 (C + 1)$).

First, we check that $a > b$:
\begin{align*}
   R + \frac{(C + 1)^2}{R} - 2 (C + 1)
   &= \frac{R^2 + C^2 + 2C + 1 - 2 RC - 2 R}{R}
\\ &= \frac{(C - R)^2 + 2(C - R) + 1}{R}
\\ &= \frac{(C - R + 1)^2 }{R}
\\ &> 0 && \text{,strictly greater since $C > R$} 
\end{align*}

Next, we compute en estimate for $\sqrt{a^2 - b^2}$:
\begin{align*}
a^2 - b^2
& =  (R + \frac{(C + 1)^2}{R})^2 - (2 (C + 1))^2\\
& =  R^2 + \frac{(C+1)^4}{R^2} + 2 (C+1)^2 - 4 (C + 1)^2 \\
& =  \left( \frac{(C + 1)^2}{R} - R \right)^2 \\
& =  \left( \frac{(C + 1)^2 - R^2}{R}\right)^2 \\
& =  \left( \frac{(C + 1 - R) (C + 1 + R)}{R}\right)^2 && \text{,since $C > R > 0$} \\
&\ge \left( \frac{(1) (R + R)}{R}\right)^2  \\
&\ge 4
\end{align*}

Finally, since $\frac{2 \pi}{\sqrt{a^2 - b^2}} \le \frac{2 \pi}{2} = \pi$, and we just proved that $\norm{g}$ is bounded by a constant. 


\subsubsection{Estimating $\int \abs{f(t)}^2$}
Function $s(k)$ (\ref{eq:bundle_s}) is too complex for direct proof of the convergence.

We proceed as follows:

\begin{itemize}
\item Since $s(k)$ is an absolute value of the determinant of the $S$ matrix, in the upper complex plane $0 \le s(k) \le 1$ holds. From here, we immediately know that the integral is bounded from above since TODO blah blah
\item We replace $s(k)$ with a lower bound $l(k)$ such that: $0 \le l(k) \le s(k) \le 1$.
\item By appling logarithm to the expression above, we get $\ln l(k) \le \ln s(k) \le 0$, and therefore, $\ln^2 s(k) \le \ln^2 l(k)$.
\item Now, if we prove convergence for $l(k)$, this immediately proves the convergence of the original integral.
\end{itemize}

We are going to use this fact in our proof and do such function changes under the integral sign, replacing complicated function with simpler (in particular, piecewise linear) lower bounds, and then using well known facts to estimate the integral.

\subsubsection{Simplifying $s(k)$}
We build a simple estimate for $s(k)$ over the upper complex plane. This function will be independent of the complex part of its argument which will make the analysis way easier.

\begin{enumerate}
\item
  It is easy to spot that if we fix the complex part of the argument, $s(k)$ is periodic with respect to the real part of its argument.
% Since $\sin k$ and $\cos k$ are periodic w.r.t the real axis, with the period of $2 \pi$, 
\mtodo{is this step really necessary?}
\item 
  Function $s(k)$ has countably infinite set of zeros, and each of them has $\Im k = \atanh \frac{2 W}{W^2 + 1}$. For brewity, we define:

  \begin{equation*}
  \begin{aligned}
     V &= \frac{2 W}{W^2 + 1}
  \\ Z &= \atanh V
  \end{aligned}
  \end{equation*}
\item
  Notice that for each $x$, $s(x + \iu y) \le s(\iu y)$. \mtodo{How to prove? We could probably take a look at numerator and denominator in separate?} 
\end{enumerate}

Given these fact, we can build a simple lower bound for $s(x + \iu y)$:
\begin{align*}
l(x + \iu y)
   &= \abs{\frac{(W^2 + 1) \sinh y - 2 W \cosh y}{(W^2 + 1) \sinh y + 2 W \cosh y}}
\\ &= \abs{\frac{\tanh y - V}{\tanh y + V}} && \text{(since $\cosh y > 0$)}
\end{align*}

It's pretty clear that $l(k)$ has a continuum of zeros over the line $\Im k = Z$, which implies that $\ln l(k)$ has continuum number of singularities over this line. Intuitively, it shouldn't spoil the convergence of integral since the contour of integration cound intersect the singularities of the original function anyway. And indeed, we will give a rigorous argument that we still can prove convergence in this case. Since the function is independent from the real part of the argument now, for brewity, we will omit it and defince $l(y) = l(\iu y)$.
\mtodo{Plot?}

% Так как $\tanh y$ неотрицательна и растет при $y \ge 0$, мы можем избавиться от знака модуля:

% \todo{make the formula bigger}
% \todo{get rid of it??}
% \[
% l(x + \iu y)
%  = \begin{cases}
%  \frac{V - \tanh y}{\tanh y + V}, & 0 \le y \le Z \\
%  \frac{\tanh y - V}{\tanh y + V}, & y > Z 
%  \end{cases}
% \]

Now, we will simplify the integral (TODO REF) using the bound we just built:
\begin{align*}
       \int\limits_{t=0}^{2 \pi} \abs{f(t)}^2 dt
   = & \int\limits_{t = 0}^{2 \pi} \ln^2 l(R \eexp{\iu t} + \iu C) \frac{R}{\norm{R \eexp{\iu t} + \iu C + i}^2} dt
\\ = & \int\limits_{t = 0}^{2 \pi} \ln^2 l(R \sin t + C) \frac{R}{R^2 \cos^2 t + (R \sin t + C + 1)^2} dt
\end{align*}
% TODO f зависит только от Im k

First, notice that:

\begin{equation*}
\begin{aligned}
   \sin(-\pi/2 + t)   &= \sin(-\pi/2 - t) = - \cos t
\\ \cos^2(-\pi/2 + t) &= \cos^2(-\pi/2 = t) = \sin^2 t
\end{aligned}
\end{equation*}
, that is, the integrand is symmetric with respect to $y$ and for proving convergence it is enough to prove convergence on the (half circle) contour defined by $-\pi/2 \le t \le \pi/2$.

Next, we do a variable change:
\begin{equation*}
\begin{aligned}
   y         &= R \sin t + C
\\ t         &= \asin \frac{y - C}{R}
\\ dt        &= \frac{1}{\sqrt{R^2 - (C - y)^2}} dy
\\ \cos t    &= \frac{\sqrt{R^2 - (C - y)^2}}{R}
\\ y(-\pi/2) &= C - R 
\\ y(\pi/2)  &= C + R 
\end{aligned}
\end{equation*}
, and after substitution:
\begin{equation}\label{eq:int_f}
\resizebox{0.9\hsize}{!}{$
\begin{aligned}
    & \int\limits_{y = C - R}^{C + R} \ln^2 f(y) \frac{R}{R^2 \cos^2 t + (R \sin t + C + 1)^2} dy \\
=   & \int\limits_{y = C - R}^{C + R} \ln^2 f(y) \frac{R}{R^2 - (C - y)^2 + (y + 1)^2} \frac{1}{\sqrt{R^2 - (C - y)^2}} dy\\
=   & \int\limits_{y = C - R}^{C + R} \ln^2 f(y) \frac{R}{R^2 - C^2 + 2 (C + 1) y + 1} \frac{1}{\sqrt{R + C - y}} \frac{1}{\sqrt{R - C + y}}  dy\\
\end{aligned}
$}
\end{equation}

On the integration path we cross singularities at $y = C - R$, $y = Z$, $y = C + R$, and the integrand is still too complex for direct estimation. Let's split the integration path in multiple segments and estimate them independently.

\subsubsection{Case 1: $C - R \le y \le Z$}
A crucial property of $s(k)$ is its convergence to 1 as $\Im k$ converges to TODO ZERO??, since then $\ln s(k)$ goes to zero, which is necessary to compensate the singularity of $\frac{1}{R - C + y}$ at $y = C - R$. 
% TODO: Интуитивно, контур интегрирования становится все ближе и ближе к $\bbR$ при увеличении радиуса и если бы $\ln s(k)$ не занулялась, сходимость бы пропала.
This implies that lower bound for $s(k)$ should also converge to $1$ as $\Im k$ goes to $0$.

Since $l$ is convex when $0 \le y \le Z$, we can estimate it from below using its first derivative at $0$. However, this is not enough, since it will this will violate the property of the estimate being positive (it is under the logarithm sign) TODO PLOT IT, so we have to be more careful and approximate the function at $Z$ with its first derivative as well, and then glue them together at some point $Z_0$:
\begin{align*}
f(y)
& = 
\begin{cases}
l'(0) y + 1   &, 0 \le y < Z_0  \\
l'(Z) (y - Z) &, Z_0 \le y < Z \\
\end{cases}
\\
& =
\begin{cases}
\frac{-2}{V} y + 1   &, 0 \le y < Z_0  \\
\frac{1}{2 V}(V^2 - 1) (y - Z) &, Z_0 \le y < Z \\
\end{cases}
\end{align*}

Note that lower bound does not have to be a continuous function, so we can pick any number in $[C - R, Z)$ as $Z_0$. For the ease of calculations, let's take $Z_0 = \frac{V}{4}$. \mtodo{plot?}

Next we investigate integrals on $[C - R, \frac{V}{4})$ and $[\frac{V}{4}, Z)$ separately.

\subsubsection{$C - R \le y < \frac{V}{4}$}

Let's compute an upper bound for the integrand (\ref{eq:int_f}). Note that for $C - R \le y \le \frac{V}{4}$:
\begin{align*}
       & \ln^2 l(y) \frac{R}{R^2 - C^2 + 2 (C + 1) y + 1} \frac{1}{\sqrt{R + C - y}} \frac{1}{\sqrt{R - C + y}}
\\ \le & \ln^2 (1 - \frac{2}{V} y) \frac{R}{R^2 - C^2 + 2 (C + 1) y + 1} \frac{1}{\sqrt{R + C - Z}} \frac{1}{\sqrt{R - C + y}}
% \\ \le & \ln^2 (1 - \frac{2}{V} y) \frac{R}{R^2 - C^2 + 2 (C + 1) y + 1} \frac{1}{\sqrt{R + C - y}} \frac{1}{\sqrt{R - C + Z}}
\end{align*}

Now, we use a well known inequality which holds for $x > -1$: TODO REFERENCE
\[
\frac{x}{1 + x} \le \ln (1 + x)
\]
, where in our case $x = -\frac{2}{V} y$. Since the expression under logarithm is less than 1, we get $\ln^2 (1 + x) \le \frac{x^2}{(1 + x)^2}$:
\begin{align*}
       & \dots 
\\ \le & \frac{\frac{4}{V^2}y^2}{(1 - \frac{2}{V}y)^2}  \frac{R}{R^2 - C^2 + 2 (C + 1) y + 1} \frac{1}{\sqrt{R + C - Z}} \frac{1}{\sqrt{R - C + y}}
\\ \le & \frac{\frac{4}{V^2}y^2}{(1 - \frac{2}{V} \frac{V}{4})^2}  \frac{R}{R^2 - C^2 + 2 (C + 1) y + 1} \frac{1}{\sqrt{R + C - Z}} \frac{1}{\sqrt{R - C + y}}
\\ = & \frac{\frac{4}{V^2}}{(1 - \frac{2}{V} \frac{V}{4})^2} \frac{R}{\sqrt{R + C - Z}} \frac{y^2}{R^2 - C^2 + 2 (C + 1) y + 1}  \frac{1}{\sqrt{R - C + y}}
\end{align*}

Note that we can ignore the terms dependent on $V$ only since $V$ is a constant independent of $r$. Now the integrand (TODO ACTUALLY ONLY ONE PART OF IT) has the form $\frac{y^2}{a + b y}$, where $a = R^2 - C^2 + 1, b = 2 (C + 1)$; when $R$ and $C$ are big enough, $a > 0$ (TODO WHY) и $b > 0$. That means $\frac{y^2}{a + b y}$ will be nonnegative when $y > 0$, and strictly increasing, since:
\[
  \left(\frac{y^2}{a + b y}\right)'
= \frac{2y}{a + by} + y^2 \frac{-b}{(a + by)^2}
= \frac{2y (a + by) - b y^2}{(a + by)^2}
= \frac{y (2a + by)}{(a + by)^2}
\ge 0
\]

This implies that we can estimate the function from above by using its value on the rightmost point of the interval at $\frac{V}{4}$.
\[
\frac{y^2}{a + b y} = \mcO\left(\frac{1}{R}\right)
\]
TODO ELABORATE ON THAT!

The remaning part of integral is:
\[
\int\limits_{C - R}^{\frac{V}{4}} \frac{1}{\sqrt{R - C + y}} = 2 \sqrt{\frac{V}{4} - C + R} = \mcO(1)
\]

And finally, after combining all these estimages, we get $\mcO(1) \mcO(\sqrt{R}) \mcO\left(\frac{1}{R}\right) \mcO(1)  = \mcO\left(\frac{1}{\sqrt{R}}\right)$ TODO EXPAND IN TERMS OF small r!

\subsubsection{$\frac{V}{4} \le y < Z$}

Let's estimate the integral (\ref{eq:int_f}) from above. Note that for $\frac{V}{4} \le y < Z$:
\begin{align*}
       & \int \ln^2 f(y) \frac{R}{R^2 - C^2 + 2 (C + 1) y + 1} \frac{1}{\sqrt{R + C - y}} \frac{1}{\sqrt{R - C + y}} 
\\ \le & \int \ln^2 f(y) \frac{R}{R^2 - C^2 + 2 (C + 1) \frac{V}{4} + 1} \frac{1}{\sqrt{R + C - Z}} \frac{1}{\sqrt{R - C + \frac{V}{4}}}
\\ \le &  \frac{R}{R^2 - C^2 + 2 (C + 1) \frac{V}{4} + 1} \frac{1}{\sqrt{R + C - Z}} \frac{1}{\sqrt{R - C + \frac{V}{4}}} \int \ln^2 \left( \frac{1}{2 V}(V^2 - 1) (y - Z) \right)
\end{align*}

$\ln^2(x)$ is integrable in the neighborhood of $x = 0$, and it is known that: %  REFTODO

\[
\int\limits_{x=x_0}^b \ln^2(a (x - b)) dx = (b - x_0) (\ln^2(a (x_0 - b)) - 2 \ln(a (x_0 - b)) + 2)
\]

In our case $x_0 = \frac{V}{4}$, $a = \frac{1}{2 V}(V^2 - 1)$, $b = Z$, and we can see that the integral is bounded by a constant independent of $R$ and $C$. Таким образом, поведение интеграла зависит только от поведения множителя:

\[
\frac{R}{R^2 - C^2 + 2 (C + 1) \frac{V}{4} + 1} \frac{1}{\sqrt{R + C - Z}} \frac{1}{\sqrt{R - C + \frac{V}{4}}} 
\]

Clearly, as $r \to \infty$, we get $R \to \infty$, $C \to \infty$, $R - C \to 0$, and we can see that the expression grows as $\mcO(\frac{1}{\sqrt{R}})$ TODO ELABORATE

\subsubsection{Case 2: $y > Z$}
When $y > Z$, 
\[
l(x + \iu y) 
 = \frac{\tanh y - V}{\tanh y + V}
 = 1 - 2 \frac{V}{\tanh y + V}
\]

% https://proofwiki.org/wiki/Inverse_of_Strictly_Increasing_Strictly_Concave_Real_Function_is_Strictly_Convex
\mtodo{proof? just take a look at second derivative?}
Since $l$ is concave, strictly increasing and for a fixed $r$, we are only interested at $y \le C + R$, we can estimate the function by a linear one: \mtodo{plot?}
\[
f(y) = \frac{l(C + R)}{C + R - Z} (y - Z)
\]
\mtodo{actually, even C + R is okay but whatever}

Notice that when $y = C + R$, for a sufficiently big $r$, $l(y)$ is bounded from both sides by constants, independent from $C$ and $R$. This is because $\tanh y$ goes to $1$ as $y$ goes to infinity, therefore, $l(y)$ in the limit will be equal to $1 - \frac{2 V}{V + 1}$. This is an important property of this specific model, since if $l$ went to zero on infinity, this part of proof couldn't have been done (see \autoref{sec:ring}).

First, not that on $Z \le y \le R + C$ the function $\frac{1}{\sqrt{R - C + y}}$ has no singularities and we can estimate integral (\ref{eq:int_f}) as:
\begin{equation}\label{eq:int_f_up}
\begin{aligned}
       & \int \ln^2 f(y) \frac{R}{R^2 - C^2 + 2 (C + 1) y + 1} \frac{1}{\sqrt{R + C - y}} \frac{1}{\sqrt{R - C + y}}
\\ \le & \int \ln^2 f(y) \frac{R}{R^2 - C^2 + 2 (C + 1) y + 1} \frac{1}{\sqrt{R + C - y}} \frac{1}{\sqrt{R - C + Z}} 
\end{aligned}
\end{equation}

To estimate this expression, let's split the integration interval in three: $[Z, Z + \frac{1}{R})$, $[Z + \frac{1}{R}, C)$, $[C, C + R]$.

\subsubsection{$Z \le y < Z + \frac{1}{R}$}

Let's estimate (\ref{eq:int_f_up}), replacing functions in integrands by their extreme value at interval's ends:
\begin{align*}
       & \int \ln^2 f(y) \frac{R}{R^2 - C^2 + 2 (C + 1) y + 1} \frac{1}{\sqrt{R + C - y}} \frac{1}{\sqrt{R - C + Z}}
\\ \le & \int \ln^2 f(y) \frac{R}{R^2 - C^2 + 2 (C + 1) Z + 1} \frac{1}{\sqrt{R + C - (Z + \frac{1}{R})}} \frac{1}{\sqrt{R - C + Z}}
\\ =   & \frac{1}{\sqrt{R + C - (Z + \frac{1}{R})}} \frac{1}{\sqrt{R - C + Z}} \frac{R}{R^2 - C^2 + 2 (C + 1) Z + 1} \int \ln^2 f(y) 
\end{align*}

\mtodo{reference?}
$\int\limits_{Z}^{Z + \frac{1}{R}} \ln^2 f(y) dy$ can be computed explicitly, using:
\[
    \int\limits_b^{b + c} \ln^2 (a (x - b)) dx = c (\ln^2(a c) - 2 \ln (a c) + 2)
\]

Hence, $\int\limits_{Z}^{Z + \frac{1}{R}} \ln^2 f(y) dy = \frac{1}{R} ( \ln^2 (\frac{l(C + R)}{C + R - Z} \frac{1}{R}) - 2 \ln (\frac{l(C + R)}{C + R - Z} \frac{1}{R}) + 2)$. As we noted above, $l(C + R)$ is bounded by constants independent on $r$, and we can clearly see that the expression grows as $\mcO(\frac{\ln^2 R}{R})$. Since the coefficient before the integral:
\[
\frac{R}{R^2 - C^2 + 2 (C + 1) Z + 1} \frac{1}{\sqrt{R + C - (Z + \frac{1}{R})}} \frac{1}{\sqrt{R - C + Z}}
\]
, has order of growth $\mcO\left(\frac{1}{\sqrt{R}}\right)$, it is clear that the integral on this interval goes to zero as $R, C \to \infty$.

\subsubsection{$Z + \frac{1}{R} \le y < C$}
Let's estimate (\ref{eq:int_f_up}) from above TODO again by picking extreme values blah blah:
\begin{align*}
       & \int \ln^2 f(y) \frac{R}{R^2 - C^2 + 2 (C + 1) y + 1} \frac{1}{\sqrt{R + C - y}} \frac{1}{\sqrt{R - C + Z}}
\\ \le & \int \ln^2 f(Z + \frac{1}{R}) \frac{R}{R^2 - C^2 + 2 (C + 1) y + 1} \frac{1}{\sqrt{R + C - C}} \frac{1}{\sqrt{R - C + Z}}
\\  =  & \ln^2 f(Z + \frac{1}{R})  \frac{R}{\sqrt{R + C - C}} \frac{1}{\sqrt{R - C + Z}} \int \frac{1}{R^2 - C^2 + 2 (C + 1) y + 1}
\end{align*}

TODO REPLACE R BY C(R) EVERYWHERE?

Antiderivative for such a function is well known:
\[
\int \frac{1}{a x + b} = \frac{\ln (a x + b)}{a}
\]
, in our case $a = 2 (C + 1)$, $b = R^2 - C^2 + 1$. Substituting the ends of the interval $(Z + \frac{1}{R}, C)$, we get the definite integral:
\[
\frac{\ln \frac{2 (C + 1) C + R^2 - C^2 + 1}{2 (C + 1) (Z + \frac{1}{R}) + R^2 - C^2 + 1}}{2 (C + 1)} = \mcO\left( \frac{\ln R}{R} \right)
\]

As $R$ goes to infinity, $\ln^2 f(Z + \frac{1}{R})$ goes to infinity as well, so we have to compute the order of singularity. Since $\ln f(Z + \frac{1}{R}) = \ln \left( \frac{l(C + R)}{C + R - Z} \frac{1}{R} \right) = \ln l(C + R) - \ln (C + R - Z) - \ln R$, we can see that $\ln^2 f(Z + \frac{1}{R}) = \mcO (\ln^2 R)$.

Finally, after combining all estimates, we get: $\mcO (\ln^2 R) \mcO\left( \frac{\ln R}{R} \right) \frac{R}{\sqrt{R}} \frac{1}{\sqrt{R - C + Z}} = \mcO\left( \frac{\ln^3 R}{\sqrt{R}} \right)$, which means that integral goes to $0$ on this interval as $r$ goes to $1$.

\subsubsection{$C \le y \le C + R$}

Оценим сверху выражение (\ref{eq:int_f_up}):
\begin{align*}
       & \ln^2 f(y) \frac{R}{R^2 - C^2 + 2 (C + 1) y + 1} \frac{1}{\sqrt{R + C - y}} \frac{1}{\sqrt{R - C + Z}}
\\ \le & \ln^2 f(C) \frac{R}{R^2 - C^2 + 2 (C + 1) C + 1} \frac{1}{\sqrt{R + C - y}} \frac{1}{\sqrt{R - C + Z}}
% \\  = & \ln^2 f(y) \mcO()
\end{align*}

Интеграл $\frac{1}{\sqrt{R + C - y}}$ от $C$ до $C + R$ тривиален и равен $2 \sqrt{R}$. Функция $f$ в точке $C$ ограничена константами не зависящими от $R$ и $C$, следовательно, и квадрат ее логарифма. Очевидно, что в результате получим выражение $\mcO \left( \frac{1}{R} \right) \sqrt R = \mcO \left( \frac{1}{\sqrt{R}} \right)$.

\subsubsection{Оценка $\int \abs{f(t)}^2$: итог}
Итого, мы разбили интегрирование на пять различных участков, и в итоге имеем порядок интеграла:
\[
\mcO \left( \frac{1}{R} \right) + \mcO \left( \frac{1}{\sqrt{R}} \right) + \mcO \left( \frac{1}{\sqrt{R}} \right) + \mcO\left( \frac{\ln^3 R}{\sqrt{R}} \right) + \mcO \left( \frac{1}{\sqrt{R}} \right) = \mcO\left( \frac{\ln^3 R}{\sqrt{R}} \right)
\]
, что означает, что весь интеграл уходит в 0 при стремлении $r$ к 1 (соответственно, $R$ к бесконечности), что и требовалось доказать.
