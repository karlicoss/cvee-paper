\section{Definitions}
\begin{itemize}
\item $\bbC$: complex plane, $\bbC = \{ x + \iu y \mid x, y \in \bbR \}$ 
\item $\bbH$: upper complex half-plane, $\bbH = \{ x + \iu y \mid y > 0, x, y \in \bbR \}$
\item $\bbD$: unit disk, $\bbD = \{ z \mid \abs{z} < 1 \}$
\item $\bbT$: unit circle, $\bbT = \partial \bbD =  \{z \mid \abs{z} = 1 \}$
\item $z$ denotes a complex argument on upper complex half-plane $\bbH$
\item $\zeta$ denotes a complex argument on unit disk $\bbD$
\end{itemize}

\section{Schrodinger's equation}
In this paper we are only interested in the mathematical scattering problem and not actual energies, so we define Plank's constant as $1$ and use stationary Schrodinger's equation: $H \Psi = E \Psi$. We are going to be interested in scattering state solutions for this equation.

\section{S-matrix}\label{sec:smatrix}
Consider a localized one dimensional potential barrier or resonator. From the left and right we TODO BLAH BLAH
Рассмотрим локализованный одномерный потенциальный барьер или резонатор. Пусть на барьер слева и справа направлены частицы с волновым вектором $k$ (который будет скаляром в одномерном случае). Слева и справа от резонатора частицы ведут себя как свободные, соответственно, в общем виде их волновые функции имеют следующий вид:

\begin{equation}\label{eq:wlwl}
\begin{aligned}
   \psi_L(x) &= A \eexp{\iu k x} + B \eexp{-\iu k x}
\\ \psi_R(x) &= C \eexp{\iu k x} + D \eexp{-\iu k x}
\end{aligned}
\end{equation}

S-matrix, or scattering matrix relates the final and initial states of the system:
\begin{equation}\label{eq:smatrix}
\begin{pmatrix} B \\ C \end{pmatrix} = S \begin{pmatrix} A \\ D \end{pmatrix}
\end{equation}
, and defines scattering properties of a potential barriers.

\section{Cayley transform}

Cayley transform maps $\bbH$ to $\bbD$:
\begin{equation}\label{eq:cayley}
W(z) = \frac{z - \iu}{z + \iu}
\end{equation}
, inverse Cayley transform maps $\bbD$ to $\bbH$:
\begin{equation}\label{eq:cayley_inverse}
w(\zeta) = \iu \frac{1 + \zeta}{1 - \zeta}
\end{equation}

One notable property of the Cayley transform is that it injectively maps $\bbR$ into unit circle $\bbT$.

Another important property we are going to use is that Cayley transform preserves circles. In particular, a circle with radius $r < 1$ centered in zero, under inverse Cayley transform maps into a circle with center $C(r)$ and radius $R(r)$, where:

\begin{equation}\label{eq:c_and_r}
\begin{aligned}
   C(r) &= \Im \frac{w(r) + w(-r)}{2}
\\ R(r) &= \Im \frac{w(r) - w(-r)}{2}
\end{aligned}
\end{equation}

From these formulas it's easy to see that if we limit $r$ to $1$, $R(r)$ goes to infinity and $C(r)$ converges to $R(r)$, which intuinively makes sense since a complex unit circle maps to a real axis. TODO ???

\section{???}
There is a connection between the S-matrix and the completeness of the resonant states for a scattering problem:

TODO something about dissipating operator

\begin{theorem}[{\cite[p. 95]{nikol2012treatise}}, {\cite[p. 99]{nikol2012treatise}}]
The following statements are equivalent:
\begin{itemize}
\item the dissipating operator $Z$ is complete
\item
\begin{equation}\label{eq:blaschke}
\lim\limits_{r = 1} \int\limits_{\bbT} \log \abs{\det S(r \zeta)} d m(\zeta) = 0
\end{equation}
\end{itemize}
\end{theorem}

\section{Convergence criterion}

Next, we will use the criterion \ref{eq:blaschke} to analyse the completeness of the system of resonant states. In the space of the unit disk it looks like:
\begin{equation}\label{eq:crit_cayley}
\lim\limits_{r = 1} \int\limits_{\abs{\zeta} = r} \log \abs{\det S(\zeta)} d \zeta = 0
\end{equation}

Since S-matrix is naturally defined on the complex plance, it makes sense to use the upper complex plane for the analysis of completeness. We change the variable in (\ref{eq:crit_cayley}), applying Cayley transform to the integral, which results in:

\begin{equation}\label{eq:crit}
\lim\limits_{r \to 1} \int\limits_{C_r} \ln \abs{\det S(k)} \frac{2 \iu}{(k + i)^2} dk = 0
\end{equation}
, where $C_r$ is an image of $\abs{\zeta} = r$ under the inverse Cayley transform. It can be parameterised as $C_r = \{R(r) \eexp{\iu t} + \iu C(r) \mid t \in [0, 2 \pi)\}$ (see \ref{eq:c_and_r}). For brewity, define:

\[
s(k) = \abs{\det S(k)}
\]
, and after throwing away constants which are irrelevant for convergence, we get the final form of the criterion, which we are going to use afterwards:

\begin{equation}\label{eq:critp}
\lim\limits_{r \to 1} \int\limits_{0}^{2 \pi} \ln s(R(r) \eexp{\iu t} + \iu C(r)) \frac{R}{(R(r) \eexp{\iu t} + \iu C(r) + i)^2} dt = 0
\end{equation}
\mtodo{$C + R = \Im w(r) = \frac{1 + r}{1 - r}$, $C - R = \Im w(-r) = \frac{1 - r}{1 + r}$}
